\documentclass{tufte-handout}

%\geometry{showframe}% for debugging purposes -- displays the margins

\usepackage{amsmath}
\usepackage{natbib}
\usepackage{bibentry}
% Set up the images/graphics package
\usepackage{graphicx}
\setkeys{Gin}{width=\linewidth,totalheight=\textheight,keepaspectratio}
\graphicspath{{graphics/}}

\title{Annotated Bibliography}
\author[Ale Sanchez-Rios]{Ale Sanchez-Rios}
% \date{24 January 2009}  % if the \date{} command is left out, the current date will be used

% The following package makes prettier tables.  We're all about the bling!
\usepackage{booktabs}

% The units package provides nice, non-stacked fractions and better spacing
% for units.
\usepackage{units}

% The fancyvrb package lets us customize the formatting of verbatim
% environments.  We use a slightly smaller font.
\usepackage{fancyvrb}
\fvset{fontsize=\normalsize}

% Small sections of multiple columns
\usepackage{multicol}

% Provides paragraphs of dummy text
\usepackage{lipsum}

% These commands are used to pretty-print LaTeX commands
\newcommand{\doccmd}[1]{\texttt{\textbackslash#1}}% command name -- adds backslash automatically
\newcommand{\docopt}[1]{\ensuremath{\langle}\textrm{\textit{#1}}\ensuremath{\rangle}}% optional command argument
\newcommand{\docarg}[1]{\textrm{\textit{#1}}}% (required) command argument
\newenvironment{docspec}{\begin{quote}\noindent}{\end{quote}}% command specification environment
\newcommand{\docenv}[1]{\textsf{#1}}% environment name
\newcommand{\docpkg}[1]{\texttt{#1}}% package name
\newcommand{\doccls}[1]{\texttt{#1}}% document class name
\newcommand{\docclsopt}[1]{\texttt{#1}}% document class option name

\begin{document}

\maketitle% this prints the handout title, author, and date

\begin{abstract}
\noindent The main subject of the paper is exploring the different ways a women of color health is impacted in pursuing a career in geosciences .
\end{abstract}

%\printclassoptions
% mea emae maedfadfjkad

\newthought{This paper will address} the reality of women in sciences as geophysics, oceanography and atmospheric sciences that are some times named as physical sciences (cite from NSF). I will use a social juste frame work to descrbie the overal impact in a women wellbeing and close look at the intersectionalliy of gender and race, to explain the current culture and the main challenges this area faces to become a more inclusive and healthier enviorment. 

% \linebreak
 \section{Reference list}
%  \citep{Bonatti2012OceanographyChallenges}
% \paragraph{1)}
%  \bibentry{Bonatti2012OceanographyChallenges}
   
%   \paragraph{new thoght}
% \paragraph{2)}
% % \cite{Englander2012DoingEBSCOhost}
%  \bibentry{Englander2012DoingEBSCOhost}
   
%   \paragraph{new thoght}
   
% 


\paragraph{1)}
% \cite{Englander2012DoingEBSCOhost}
 \bibentry{Englander2012DoingEBSCOhost}
   
   \paragraph{
This is one of the few papers that trully addreses how culture upbringing affects women perceptions of herself as a scientist and and set a tone for her interactions with others. The authors are from Baja California, Mexico, and address the issues of Machismo in Mexican culture and how it  translate in the physical sciences. They also explore another Mexican cultural behavior of Marianismo, wish comes from the word Maria. Is refers to the idea that women have be a submissive, a `good girl', and discuss how this affects women in science. It also makes a summary of how other cultures construct the image of a women and why this is important to acknowlege when we talk of women in sciences} 

\paragraph{2)}
%  \cite{Gay-Antaki2018ClimateChange}
 \bibentry{Gay-Antaki2018ClimateChange}
   
   \paragraph{
 The relevance of this paper is that they acknowledge the importance of intersectionality in the experiences of women in the creation of the Intergovernmental Panel of Climate Change (IPCC), which is one of the most important studies that analyzes the impact humans had had in the Climate change, and the overall health of the planet. With an intersectional lens, they also compare the amount of authors that come from Global North and Global South, and realize how this could impact the interpretation of the results of the study. This study offers an excellent examples of how bias can affect the results of a very important study}

\paragraph{3)}
%   \cite{Cabay2018ChillyPrograms}
 \bibentry{Cabay2018ChillyPrograms}
   
   \paragraph{
The authors are not from the geophysical sciences
This paper analysis the lives of 28 women students studying for a Ph.D., and analyzes  how they are making decisions for their future careers, relating it to the `licky pipe problem'. They argue that instead of leaving academia it should be seen that women are choosing to change paths from a research base work to a more “altruistic” or governmental job. I do not completely agree with the conclusions of this paper, as they do not address the emotional problems that pushes a women out of academia, and does not acknowledge different backgrounds, or different time frames, and assumes all students go through the same thinking process at the same time in their carrers.} 

\paragraph{4)}
%   \cite{ClancyDoubleHarassment}
 \bibentry{ClancyDoubleHarassment}
   
   \paragraph{
This paper is both a great source of “data” to does that need numbers to understand a problem, and is also heartbreaking, as it shines light to the reality of many women of color in planetary sciences that had to do field work and that have to deal with a power dynamics that is unique. This paper is one of the few in this discipline (if not the only one) that validates the different struggles that a women of color will have to faced in astronomy or other planetary sciences.}

\paragraph{5)}
  \cite{Pereira2016AreSciences}
 \bibentry{Pereira2016AreSciences}
   
   \paragraph{

This paper is from an international journal,  and looks into specifics of how bias impact women scientist and they way they are seen in the academic world. It examines how number of publications and citations are calculated  ( and how themselves are bias products), and the disparity of the amount of men and women that are working as peer reviews and board members of journals. This paper is a great example of how bias can indirectly remove a women from the academic circle and undermine the importance of her science
}

\paragraph{6)}
  \cite{Pereira2016AreSciences}
 \bibentry{Pereira2016AreSciences}
   
   \paragraph{
5) A Louis, and istele (2011) The Differences in scoresand self- efficacy
This paper does a great job in exploring the impact in students of positive reinforcement or negative comments, and how it had impacted the performance of women who had been told that men are usually better than women in math, or that this test will be too hard for you, or other similar microagressions. This behaviors starts early in the K12 system, and I believe that even a `successful’ women in math or physics will carry an internalize idea that she is not that good in this disciplines becasue she been told over and over that it might be to hard for her. 


7) Oceanos
This paper gives a great background of how women started working in oceanography, more precisely, in research vessels and in the field. It gives an anecdotal review of the enviorment of the oceanography field in the ‘50s and how woemn were perceive by their male colegues. This paper helps set the tone of the ideas that has transcended 50 years of what a women is good for in sciences, and the overall interactions of men and women in research vessels.

8) Johnsn Increasing deversity by mentoring    
This paper focus expecificily in oceanogaphy and the lack of diversity in  the faculty and in the graduate student level. The authors try to trace it down to a problem of lack of mentorship, that because young scientist and students do not get the right type of advice, support or guidance  they drop out of the career. Even thouh I believe lack or representationa and ther for mentorship is a part of a problem, I do not aggree compeltey with this paper as it makes seem that then ony thing needed for a change in a an increase of better mentors in the discipline, it does not addres the culture, the bias nor the stiga related with certain personalies traits in the sciences.

9) Bendels Gender disparires
This paper does a extensive study in the how the way scientes are eing “graded” in academia is bias in gender, giving preference to men. They also take a close look to the h-index to understand if bias could affect the perception of women in the field, and explain a negative feedback between how women general well being in academia; how they are “graded” and then how people cite them or not, which then impact their funding and support from university, then their overall self-steem and then their performance, which then cause less papers and less citations. 

\bibliography{mendeley}
\bibliographystyle{newapa}



\end{document}
